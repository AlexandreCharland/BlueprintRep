\chapter{YoungTableau}

\begin{definition}[YoungTableau]
    \label{YoungTableau}
    \lean{YoungTableau}
    \leanok
    Un YoungTableau est une fonction des cellules d'un YoungDiagram de taille n et retourne un naturel de 0 à n-1
\end{definition}

\begin{lemma}[injYu]
    \label{injYu}
    \lean{injYu}
    \leanok
    \uses{YoungTableau}
    Un YoungTableau est injectif sur les entrés qui sont dans le YoungDiagram
\end{lemma}
\begin{proof}
    \leanok
    Par définition d'un YoungTableau
\end{proof}

\begin{lemma}[bijYu]
    \label{bijYu}
    \lean{bijYu}
    \leanok
    Un YoungTableau est une bijection entre les case de son YoungDiagram et les naturels de 0 à n-1
\end{lemma}
\begin{proof}
    \uses{injYu}
    \leanok
    Comme il est injectif et le domaine et codomaine sont fini et ont la même cardinalité.\\
    La fonction doit être bijective
\end{proof}

\begin{lemma}[preImYu]
    \label{preImYu}
    \lean{preImYu}
    \leanok
    Tous nombre de 0 à n-1 possède une unique case associé dans $\mu$ par $Y_{\mu}$
\end{lemma}
\begin{proof}
    \uses{bijYu}
    \leanok
    Trivial sachant que $Y_{\mu}$ est bijectif
\end{proof}

\begin{definition}[Pu]
    \label{Pu}
    \lean{Pu}
    \leanok
    $P_{\mu}$ est un sous groupe de $S_{n}$, défini de la façon suivante:\\
    Un élément de $P_{\mu}$ permute les entré du YoungDiagram si ils sont sur la même rangé.
\end{definition}
\begin{proof}
    \uses{injYu, preImYu}
    \leanok
    Il y a trois choses à vérifier.\\
    Le sous-groupe est fermé sous la composition de fonction\\
    Preuve:\\
    Soit $\alpha, \beta \in P_{\mu}$, mq $\alpha \circ \beta$($Y_{\mu}$(i)) = $Y_{\mu}$(j) $\to$ i.y = j.y\\
    Comme $Y_{\mu}$ est une bijection, $\exists k \in \mu$ tq $Y_{\mu}$(k) = $\beta$($Y_{\mu}$(j))\
    Comme $\beta \in P_{\mu}$ on a que k.y = j.y\\
    De plus on a que $\alpha (Y_{\mu}$(k)) = $\alpha \circ \beta$($Y_{\mu}$(i)) = $Y_{\mu}$(j)\\
    On peut déduire que i.y = k.y = j.y\\
    \\
    L'élement neutre est élément de $P_{\mu}$\\
    La preuve découle de l'injectivité de $Y_{\mu}$\\
    \\
    L'inverse est élément de $P_{\mu}$\\
    Soit $\alpha \in P_{\mu}$, mq $\alpha^{-1} \in P_{\mu}$\\
    Comme alpha est une bijection, on a que $\alpha^{-1} (Y_{\mu}$(i)) = $Y_{\mu}$(j) $\Leftrightarrow$ $Y_{\mu}$(i) = $\alpha (Y_{\mu}$(j))
\end{proof}

\begin{definition}[PuCard]
    \uses{Pu}
    \label{PuCard}
    \lean{PuCard}
    \leanok
    Le nombre d'élément de $P_{\mu}$ est fini.
\end{definition}
\begin{proof}
    \leanok
    Comme $P_{\mu}$ est un sous-groupe d'un groupe fini, il a un nombre fini d'élément.
\end{proof}

\begin{definition}[Qu]
    \label{Qu}
    \lean{Qu}
    \leanok
    $Q_{\mu}$ est un sous groupe de $S_{n}$, défini de la façon suivante:\\
    Un élément de $Q_{\mu}$ permute les entré du YoungDiagram si ils sont sur la même colonne.
\end{definition}
\begin{proof}
    \uses{injYu, preImYu}
    \leanok
    La même preuve que Pu
\end{proof}

\begin{definition}[QuCard]
    \uses{Qu}
    \label{QuCard}
    \lean{QuCard}
    \leanok
    Le nombre d'élément de $Q_{\mu}$ est fini.
\end{definition}
\begin{proof}
    \leanok
    Comme $Q_{\mu}$ est un sous-groupe d'un groupe fini, il a un nombre fini d'élément.
\end{proof}

\begin{lemma}[sectPuQu]
    \label{sectPuQu}
    \lean{sectPuQu}
    \leanok
    \uses{Pu, Qu}
    Pour un même YoungTableau, l'intersection de $P_{\mu}$ et $Q_{\mu}$ est ${1}$
\end{lemma}
\begin{proof}
    \leanok
    Il faut mq $P_{\mu} \cap Q_{\mu} \subseteq {1}$\\
    Soit $\alpha \in P_{\mu} \cap Q_{\mu}$ et i $\in \mu$\\
    Comme $Y_{\mu}$ est bijectif, $\exists$ j $\in \mu$, $\alpha ( Y_{\mu} (i)) = Y_{\mu} (j)$\\
    $\alpha \in P_{\mu} \cap Q_{\mu}$ donc i.x = j.x et i.y = j.y\\
    Donc i=j $\to \alpha(Y_{\mu}(i))=Y_{\mu}(i)$\\
    Donc $alpha$ est la fonction id.\
\end{proof}

\begin{definition}[PuQu]
    \label{PuQu}
    \uses{Pu, Qu}
    \lean{PuQu}
    \leanok
    $P_{\mu}Q_{\mu} := \{g : [0,n-1] \to [0,n-1] | \exists p \in P_{\mu} \land \exists q \in Q_{\mu}, g = p q \}$
\end{definition}

\begin{definition}[Gu]
    \label{Gu}
    \lean{Gu}
    \leanok
    \uses{YoungTableau}
    $G_{\mu}$ est une permutation de [0, n-1] tq
    \[ \forall i,j,k,l \in \mu, ((i \neq j) \land (G_{\mu} \circ Y_{\mu} (i) = Y_{\mu} (k)) \land (G_{\mu} \circ Y_{\mu} (j) = Y_{\mu} (l))) → ((i.x \neq j.x) \lor (k.y \neq l.y)) \]
\end{definition}

\begin{definition}[YuInv]
    \label{YuInv}
    \lean{YuInv}
    \leanok
    \uses{YoungTableau, bijYu}
    $Y_{\mu}^{-1}$ est une l'inverse de $Y_{\mu}$
\end{definition}

\begin{lemma}[staysInY]
    \label{staysInY}
    \lean{staysInY}
    \leanok
    \uses{Gu, YuInv}
    \[ \forall m \in [0,n-1], (Y_{\mu}^{-1}(m).x,Y_{\mu}^{-1}(G_{\mu}(m)).y) \in \mu \]
\end{lemma}
\begin{proof}
    No idea...
    TODO Figure it out
\end{proof}

\begin{definition}[qu]
    \label{qu}
    \lean{qu}
    \leanok
    \uses{staysInY}
    $q_{\mu}$ est une permutation de [0, n-1] défini comme
    \[ q_{\mu}(m)= Y_{\mu}((Y_{\mu}^{-1}(m)).x,(Y_{\mu}^{-1} \circ G_{\mu}(m)).y) \]
\end{definition}
\begin{proof}
    \leanok
    Par le lemme staysInY, on sait que la fonction $q_{\mu}$ est bien défini.\\
    Il ne reste plus qu'a montré que $q_{\mu}$ est une bijection.\\
    TODO
\end{proof}

\begin{definition}[quInv]
    \label{quInv}
    \lean{quInv}
    \leanok
    \uses{qu}
    $q_{\mu}^{-1}$ est la fonction inverse de $q_{\mu}$
\end{definition}

\begin{lemma}[staysInX]
    \label{staysInX}
    \uses{quInv}
    \[ \forall m \in [0,n-1], ((Y_{\mu}^{-1} \circ G_{\mu} \circ q_{\mu}^{-1}(m)).x, (Y_{\mu}^{-1}(m)).y) \in \mu \]
\end{lemma}
\begin{proof}
    No idea...
    TODO Figure it out
\end{proof}

\begin{definition}[pu]
    \label{pu}
    \uses{staysInX}
    $p_{\mu}$ est une permutation de [0,n-1] défini comme
    \[ q_{\mu}(m)= Y_{\mu}(Y_{\mu}^{-1}(m).x,Y_{\mu}^{-1}(G_{\mu}(m)).y) \]
\end{definition}
\begin{proof}
    TODO
    
\end{proof}

\begin{lemma}[No2FromSameColToSameRow]
    \label{No2FromSameColToSameRow}
    \uses{PuQu}
    Soit $g : [0,n-1] \to [0,n-1]$ une fonction bijective et $Y_{\mu}$ un YoungTableau.\\
    Si $\forall i, j, k, l \in \mu, i \neq j, g(Y_{\mu}(i)) = Y_{\mu}(k), g(Y_{\mu}(j)) = Y_{\mu}(l)$ alors $i.x \neq j.x \lor k.y \neq l.y$.\\
    Alors $g \in P_{\mu}Q_{\mu}$
\end{lemma}
\begin{proof}
    \uses{pu}
    Posons $q(Y_{\mu}(i)) := Y_{\mu}(i.x,(Y_{\mu}^{-1} \circ g \circ Y_{\mu}(i)).y)$.\\
    Par le lemme qWellDefined, nous avons que q est bien définit.\\
    Montrons que $q \in Q_{\mu}$\\
    Si q n'est pas injectif alors $\exists k, l \in \mu$ tq $k \neq l, q(Y_{\mu}(k))=q(Y_{\mu}(l))$.\\
    Donc $Y_{\mu}((Y_{\mu}^{-1} \circ g \circ Y_{\mu}(k)).x,k.y) = Y_{\mu}((Y_{\mu}^{-1} \circ g \circ Y_{\mu}(l)).x,l.y)$.\\
    Comme $Y_{\mu}^{-1} \circ g \circ Y_{\mu}$ est bijectif, $\exists! i,j \in \mu$ tq $i \neq j, Y_{\mu}^{-1} \circ g Y_{\mu}(k)=i, Y_{\mu}^{-1} \circ g Y_{\mu}(l)=j$.\\
    Donc $\exists i, j, k, l \in \mu, i \neq j, g(Y_{\mu}(i)) = Y_{\mu}(k), g(Y_{\mu}(j)) = Y_{\mu}(l)$ et $i.x = j.x \land k.y = l.y$.\\
    Contradiction d'hypothèse.\\
    Donc $q$ est injectif. De plus comme le domaine et codomaine sont finis et de même taille, on a que $q$ est une bijection. Ainsi $q \in Q_{\mu}$.\\
    \\
    Posons $p(Y_{\mu}(i)) := Y_{\mu}((Y_{\mu}^{-1} \circ g \circ q^{-1} \circ Y_{\mu}(i)).x,i.y)$.\\
    On remarque $p \circ q = g$. Soit $i \in \mu$. $\exists j \in \mu$ tq $g(Y_{\mu}(i))=Y_{\mu}(j)$.\\
    Donc $Y_{\mu}^{-1} \circ g \circ Y_{\mu}(i) = j$
    \[ p \circ q(Y_{\mu}(i)) = p (Y_{\mu}(i.x,(Y_{\mu}^{-1} \circ g \circ Y_{\mu}(i)).y)) = p (Y_{\mu}(i.x,j.y)) \]
    \[Y_{\mu}((Y_{\mu}^{-1} \circ g \circ q^{-1} \circ Y_{\mu}(i.x,j.y)).x,j.y) = Y_{\mu}((Y_{\mu}^{-1} \circ g \circ Y_{\mu}(i)).x,j.y)\]
    \[ Y_{\mu}(j.x,j.y) = Y_{\mu}(j)\]
    Donc $p$ est bien définit, et $g \in P_{\mu} Q_{\mu}$
\end{proof}

\begin{definition}[IneqYoungDiagram]
    \label{IneqYoungDiagram}
    \lean{IneqYoungDiagram}
    \leanok
    Soit $\mu$ et $\lambda$ deux YoungDiagram de même cardinalité.\\
    On dit que $\mu > \lambda$ si $\exists i \in \mathbb{N}$ tq $\mu_{i}>\lambda_{i}$ et $\forall j \in \mathbb{N}_{<i}$, $\mu_{j} = \lambda_{j}$.
\end{definition}