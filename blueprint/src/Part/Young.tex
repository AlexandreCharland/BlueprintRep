\chapter{YoungTableau}

\begin{definition}[YoungTableau]
    \label{YoungTableau}
    \lean{YoungTableau}
    \leanok
    Un YoungTableau est une fonction des cellules d'un YoungDiagram de taille n et retourne un naturel de 0 à n-1
\end{definition}

\begin{lemma}[injYu]
    \label{injYu}
    \lean{injYu}
    \leanok
    Un YoungTableau est injectif sur les entrés qui sont dans le YoungDiagram
\end{lemma}
\begin{proof}
    \uses{YoungTableau}
    \leanok
    Par définition d'un YoungTableau
\end{proof}

\begin{lemma}[bijYu]
    \label{bijYu}
    \lean{bijYu}
    \leanok
    Un YoungTableau est une bijection entre les case de son YoungDiagram et les naturels de 0 à n-1
\end{lemma}
\begin{proof}
    \uses{injYu}
    \leanok
    Comme il est injectif et le domaine et codomaine sont fini et ont la même cardinalité.\\
    La fonction doit être bijective
\end{proof}

\begin{definition}[Pu]
    \label{Pu}
    \lean{Pu}
    \leanok
    Pu est un sous groupe de $S_{n}$, défini de la façon suivante:\\
    Un élément de Pu permute les entré du YoungDiagram si ils sont sur la même rangé.
\end{definition}
\begin{proof}
    \uses{injYu, bijYu}
    \leanok
    Il y a trois choses à vérifier.\\
    Le sous-groupe est fermé sous la composition de fonction\\
    Preuve:\\
    Soit $\alpha, \beta \in P_{\mu}$, mq $\alpha \circ \beta$($Y_{\mu}$(i)) = $Y_{\mu}$(j) $\to$ i.y = j.y\\
    Comme $Y_{\mu}$ est une bijection, $\exists k \in \mu$ tq $Y_{\mu}$(k) = $\beta$($Y_{\mu}$(j))\
    Comme $\beta \in P_{\mu}$ on a que k.y = j.y\\
    De plus on a que $\alpha (Y_{\mu}$(k)) = $\alpha \circ \beta$($Y_{\mu}$(i)) = $Y_{\mu}$(j)\\
    On peut déduire que i.y = k.y = j.y\\
    \\
    L'élement neutre est élément de $P_{\mu}$\\
    La preuve découle de l'injectivité de $Y_{\mu}$\\
    \\
    L'inverse est élément de $P_{\mu}$\\
    Soit $\alpha \in P_{\mu}$, mq $\alpha^{-1} \in P_{\mu}$\\
    Comme alpha est une bijection, on a que $\alpha^{-1} (Y_{\mu}$(i)) = $Y_{\mu}$(j) $\Leftrightarrow$ $Y_{\mu}$(i) = $\alpha (Y_{\mu}$(j))
\end{proof}

\begin{definition}[Qu]
    \label{Qu}
    \lean{Qu}
    \leanok
    Pu est un sous groupe de $S_{n}$, défini de la façon suivante:\\
    Un élément de Pu permute les entré du YoungDiagram si ils sont sur la même colonne.
\end{definition}
\begin{proof}
    \uses{injYu, bijYu}
    \leanok
    La même preuve que Pu
\end{proof}

\begin{lemma}[sectPuQu]
    \label{sectPuQu}
    \lean{sectPuQu}
    \leanok
    Pour un même YoungTableau, l'intersection de $P_{\mu}$ et $Q_{\mu}$ est ${1}$
\end{lemma}
\begin{proof}
    \uses{Pu, Qu}
    \leanok
    Il faut mq $P_{\mu} \cap Q_{\mu} \subseteq {1}$\\
    Soit $\alpha \in P_{\mu} \cap Q_{\mu}$ et i $\in \mu$\\
    Comme $Y_{\mu}$ est bijectif, $\exists$ j $\in \mu$, $\alpha ( Y_{\mu} (i)) = Y_{\mu} (j)$\\
    $\alpha \in P_{\mu} \cap Q_{\mu}$ donc i.x = j.x et i.y = j.y\\
    Donc i=j $\to \alpha(Y_{\mu}(i))=Y_{\mu}(i)$\\
    Donc alpha est la fonction identité.\
\end{proof}

\begin{definition}[YoungProjectors]
    \label{YoungProjectors}
    \uses{Pu, Qu}
    Un Young projector est défini par un YoungDiagram $\mu$\\
    \[ a_{\mu} := \frac{1}{|P_{\mu}|}\sum_{g \in P_{\mu}}g \]
    \[ b_{\mu} := \frac{1}{|Q_{\mu}|}\sum_{g \in Q_{\mu}}(-1)^{g}\]
    Où $(-1)^{g}$ est le signe de g
\end{definition}

\begin{definition}[YoungSymmetriser]
    \label{YoungSymmetriser}
    \uses{YoungProjectors}
    Un Young symmetriser est défini par un YoungDiagram $\mu$\\
    $c_{\mu} := a_{\mu} b_{\mu} $
\end{definition}