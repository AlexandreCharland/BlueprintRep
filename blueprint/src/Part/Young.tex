\chapter{Specht modules}

\begin{definition}[YoungTableau]
    \label{YoungTableau}
    \lean{YoungTableau}
    \leanok
    Un YoungTableau est une fonction des cellules d'un YoungDiagram de taille n et retourne un naturel de 0 à n-1
\end{definition}

\begin{theorem}[injYu]
    \label{injYu}
    \lean{injYu}
    \leanok
    Un YoungTableau est injectif sur les entrés qui sont dans le YoungDiagram
\end{theorem}
\begin{proof}
    \uses{YoungTableau}
    \leanok
    Par définition d'un YoungTableau
\end{proof}


\begin{theorem}[bijYu]
    \label{bijYu}
    \lean{bijYu}
    \leanok
    Un YoungTableau est une bijection entre les case de son YoungDiagram et les naturels de 0 à n-1
\end{theorem}
\begin{proof}
    \uses{injYu}
    \leanok
    Comme il est injectif et le domaine et codomaine sont fini et ont la même cardinalité.\\
    La fonction doit être bijective
\end{proof}