\chapter{Module}

\begin{lemma}[HomAeM]
    \label{HomAeM}
    Soit A un anneau unitaire, M un A-module et e un idempotant de A.
    \[ \forall \phi \in \text{Hom}_{A}(Ae,M), \exists m \in e M, \phi(e)=m\]
\end{lemma}
\begin{proof}
    Soit $ n \in M$ tq $\phi(e)=n$
    \[ \phi(e) = n \Rightarrow e \phi(e) = \phi(e) = e n \Rightarrow n = e n \]
    Donc $ n \in e M$
\end{proof}

\begin{lemma}[HomAeMisoeM]
    \label{HomAeMisoeM}
    Soit A un anneau unitaire, M un A-module et e un idempotant de A.
    \[ \text{Hom}_{A}(A e,M) \cong e M\]
\end{lemma}
\begin{proof}
    \uses{HomAeM}
    Soit $\phi : \text{Hom}_{A}(A e,M) \to M$ tq $\phi(\psi) = \psi(e)$\\
    Il faut mq $\phi$ est un homomorphisme.\\
    Soit $\psi , \sigma \in \text{Hom}_{A}(A e,M)$.\\
    \[ \phi(\psi + \sigma) = (\psi + \sigma)(e) = \psi(e) + \sigma(e) = \phi(\psi) + \phi(\sigma) \]
    Soit $a \in A$
    \[ a \cdot \phi(\psi) = a \cdot \psi(e) = \phi(a \cdot \psi) \]
    Comme $\text{Hom}_{A}(A e,M)$ est un module, par le premier théorème d'isomorphisme, \\
    $\frac{\text{Hom}_{A}(A e,M)}{\text{ker}(\phi)} \cong \text{Im}(\phi)$\\
    Le noyau de $\phi$ est trivial, car un unique homomorphisme peut être défini tq $\sigma(e) = 0$, car tous éléments du domaine sont de la forme ae.\\
    Il ne reste plus qu'a montré que l'image est eM.\\
    Par le lemme HomAeM, on a que $\text{Im}(\phi) \subset e M$.\\
    Soit $m \in e M$. Considérons $\psi(e)=m$ tq $\psi(ae) = am$.\\
    Il s'agit d'un homomorphisme de $A e \to M$.\\
    Donc $\text{Im}(\phi) = e M$
\end{proof}