\chapter{Module}

\begin{lemma}[HomAMisoM]
    \label{HomAMisoM}
    Soit A un anneau unitaire et M un A-module.
    \[ \text{Hom}_{A}(A,M) \cong M \]
\end{lemma}
\begin{proof}
    Soit $m \in M$, $\psi_{m} : A \to M$ tq $\psi_{m}(a) = a m$\\
    Soit $\phi : M \to (A \to M)$ tq $\phi(m) = \psi_{m}$\\
    Il faut montré que $\psi_{m}$ est un homomorphisme.\\
    Soit $a,b \in A$.\\
    \[ \psi_{m}(a+b) = (a+b)m = a m + b m = \psi_{m}(a) + \psi_{m}(b) \]
    On a que $\phi$ est un homomorphisme, car\\
    $\phi(m+n) = \psi_{m+n}$.\\
    $\forall a \in A$, $\psi_{m+n}(a) = a (m+n) = (a m) + (a n) = \psi_{m}(a)+\psi_{n}(a)$\\
    Donc $\psi_{m+n} = \psi_{m}+\psi_{n} \Rightarrow \phi(m+n) = \phi(m)+\phi(n)$\\
    Par le premier théorème d'isomorphisme de module, $\frac{M}{\text{ker}(\phi)} \cong \text{Im}(\phi)$\\
    Seul $\phi(0)$ envoit à l'identité de $\text{Hom}_{A}(A,M)$, donc le noyau est trivial. Il ne reste plus qu'a montré que $\phi$ atteint tous les homomorphismes de A à M.\\
    Soit $\sigma \in \text{Hom}_{A}(A,M)$ et $m \in M$, tq $\sigma(1)=m$.\\
    Soit $a \in A$\\
    $\sigma(a) = \sigma(a \cdot 1) = a \cdot \sigma(1)$, car tous élément de l'algèbre agit comme un scalaire sur l'homomorphisme.\\
    Donc $\forall a \in A, \sigma(a) = a \cdot \sigma(1) = a\cdot m$.\\
    $\sigma = \psi_{m}$.\\
    Donc $\phi$ est surjectif et on obtient le résultat voulu.
\end{proof}

\begin{lemma}[MdirectSumIdemp]
    \label{MdirectSumIdemp}
    Soit A un algèbre et $e \in A$ un idempotant de A.
    \[ A = A e \oplus A (1-e) \]
\end{lemma}
\begin{proof}
    Il suffit de montré que
    $\forall m \in A e$ et $\forall n \in A (1-e)$ tq $m+n=0 \Rightarrow m=n=0$.\\
    Soit m,n tq décris plus haut.\\
    Comme $m \in A e, \exists a \in A$ tq $a e = m$.\\
    Comme $n \in A (1-e), \exists b \in A$ tq $b (1-e) = n$.\\
    \[ m+n=a e + b (1-e) = 0 \Rightarrow a e^{2} + b e - b e^{2} = a e + b e - b e = 0 e \Rightarrow m = a e = 0 \]
    \[ 0+n=0 \Rightarrow n=0\]
\end{proof}

\begin{lemma}[HomAeM]
    \label{HomAeM}
    Soit A un anneau unitaire, M un A-module et e un idempotant de A.
    \[ \forall \phi \in \text{Hom}_{A}(Ae,M), \exists m \in e M, \phi(e)=m\]
\end{lemma}
\begin{proof}
    \uses{HomAMisoM}
    Si e=0\\
    On a que A0 = {0} donc $\forall a \in A$\\
    \[ \phi(0) = \phi(a 0) = a \phi (0) \]
    On conclue que la seul valeur de $\phi(0)=0 \in$ 0M\\
    Si e=1\\
    Par le lemme HomAMisoM\\
    Sinon $\exists n \in M$ tq $\phi(e)=n$
    \[ \phi(e) = n \Rightarrow e \phi(e) = \phi(e) = e n \Rightarrow n = e n \]
    Il faut que n soit en mesure d'absorber $e \notin {0,1}$. On conclue que $n \in e M$\\
\end{proof}

\begin{lemma}[HomAeMisoeM]
    \label{HomAeMisoeM}
    Soit A un anneau unitaire, M un A-module et e un idempotant de A.
    \[ \text{Hom}_{A}(A e,M) \cong e M\]
\end{lemma}
\begin{proof}
    \uses{HomAeM}
    Soit $m \in e M$, $\psi_{m} : A e \to M$ tq $\psi_{m}(a) = a m$\\
    Il faut montré que $\psi_{m}$ est un homomorphisme.\\
    Soit $a,b \in A e$.\\
    \[ \psi_{m}(a+b) = (a+b)m = a m + b m = \psi_{m}(a) + \psi_{m}(b) \]
    Soit $\phi : e M \to (A e \to M)$ tq $\phi(m) = \psi_{m}$\\
    On a que $\phi$ est un homomorphisme, car\\
    $\phi(m+n) = \psi_{m+n}$.\\
    $\forall a \in A$, $\psi_{m+n}(a) = a (m+n) = (a m) + (a n) = \psi_{m}(a)+\psi_{n}(a)$\\
    Donc $\psi_{m+n} = \psi_{m}+\psi_{n} \Rightarrow \phi(m+n) = \phi(m)+\phi(n)$\\
    Par le premier théorème d'isomorphisme de module, $\frac{e M}{\text{ker}(\phi)} \cong \text{Im}(\phi)$\\
    Seul $\phi(0)$ envoit à l'identité de $\text{Hom}_{A}(A,M)$, donc le noyau est trivial.\\
    Par le lemme HomAeM, on a que tous les homomorphismes de $\text{Hom}_{A}(A,M)$ sont atteint.
\end{proof}