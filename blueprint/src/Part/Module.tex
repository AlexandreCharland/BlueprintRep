\chapter{Module}

\begin{lemma}[MdirectSumIdemp]
    \label{MdirectSumIdemp}
    Soit A un anneau unitaire et $e \in A$ un idempotant de A.
    \[ A = e A \oplus (1-e) A \]
\end{lemma}
\begin{proof}
    Il suffit de montré que
    $\forall m \in e A$ et $\forall n \in (1-e) A$ tq $m+n=0 \Rightarrow m=n=0$.\\
    Soit m,n tq décris plus haut.\\
    Comme $m \in e A, \exists a \in A$ tq $e a = m$.\\
    Comme $n \in (1-e) A, \exists b \in A$ tq $(1-e) b = n$.\\
    \[ m+n=e a + (1-e) b = 0\]
    \[ e \cdot (e a + (1-e) b) = e^{2} a + (e - e^{2}) b = e a = e 0 \Rightarrow m=0\]
    \[ 0+n=0 \Rightarrow n=0\]
\end{proof}

\begin{lemma}[HomAeM]
    \label{HomAeM}
    Soit A un anneau unitaire, M un A-module et e un idempotant de A.
    \[ \forall \phi \in \text{Hom}_{A}(Ae,M), \exists m \in e M, \phi(e)=m\]
\end{lemma}
\begin{proof}
    \uses{MdirectSumIdemp}
    Soit $ n \in M$ tq $\phi(e)=n$
    \[ \phi(e) = n \Rightarrow e \phi(e) = \phi(e) = e n \Rightarrow n = e n \]
    Par le lemme MdirectSumIdemp, on a que $A = e A \oplus (1-e) A$\\
    $\exists ! m,a \in A$ tq $n = e m + (1-e) b$.\\
    $e m + (1-e) b = n = e n = e \cdot (e m + (1-e) b) = e^{2} m + (e - e^{2}) b = e m$\\
    Donc $b = 0 \Rightarrow n= em \Rightarrow n \in e M$.
\end{proof}

\begin{lemma}[HomAeMisoeM]
    \label{HomAeMisoeM}
    Soit A un anneau unitaire, M un A-module et e un idempotant de A.
    \[ \text{Hom}_{A}(A e,M) \cong e M\]
\end{lemma}
\begin{proof}
    \uses{HomAeM}
    Soit $m \in e M$, $\psi_{m} : A e \to M$ tq $\psi_{m}(a) = a m$\\
    Il faut montré que $\psi_{m}$ est un homomorphisme.\\
    Soit $a,b \in A e$.\\
    \[ \psi_{m}(a+b) = (a+b)m = a m + b m = \psi_{m}(a) + \psi_{m}(b) \]
    Soit $\phi : e M \to (A e \to M)$ tq $\phi(m) = \psi_{m}$\\
    On a que $\phi$ est un homomorphisme, car\\
    $\phi(m+n) = \psi_{m+n}$.\\
    $\forall a \in A$, $\psi_{m+n}(a) = a (m+n) = (a m) + (a n) = \psi_{m}(a)+\psi_{n}(a)$\\
    Donc $\psi_{m+n} = \psi_{m}+\psi_{n} \Rightarrow \phi(m+n) = \phi(m)+\phi(n)$\\
    Par le premier théorème d'isomorphisme de module, $\frac{e M}{\text{ker}(\phi)} \cong \text{Im}(\phi)$\\
    Seul $\phi(0)$ envoit à l'identité de $\text{Hom}_{A}(A,M)$, donc le noyau est trivial.\\
    La dernière chose a montré est que l'image de $\phi$ est $\text{Hom}_{A}(A e,M)$.\\
    Soit $\sigma \in \text{Hom}_{A}(A e,M)$.\\
    Par le lemme HomAeM, $\exists m \in eM$ tq $\sigma(e) = m$.\\
    Montrons que $\sigma = \phi(m) = \psi_{m}$\\
    Soit $a \in Ae$ et $b \in A$ tq $a= b e$\\
    \[ \sigma(a) = \sigma (b e) = \sigma(b e^{2}) = b e \cdot \sigma(e) = b e m = a m = \psi_{m}(a) \]
    Donc $\sigma \in \text{Im}(\phi)$
\end{proof}

\begin{lemma}[HomAeMisoeM']
    \label{HomAeMisoeM'}
    Soit A un anneau unitaire, M un A-module et e un idempotant de A.
    \[ \text{Hom}_{A}(A e,M) \cong e M\]
\end{lemma}
\begin{proof}
    \uses{HomAeM}
    Soit $\phi : \text{Hom}_{A}(A e,M) \to M$ tq $\phi(\psi) = \psi(e)$\\
    Il faut mq $\phi$ est un homomorphisme.\\
    Soit $\psi , \sigma \in \text{Hom}_{A}(A e,M)$.\\
    \[ \phi(\psi + \sigma) = (\psi + \sigma)(e) = \psi(e) + \sigma(e) = \phi(\psi) + \phi(\sigma) \]
    Soit $a \in A$
    \[ a \cdot \phi(\psi) = a \cdot \psi(e) = \phi(a \cdot \psi) \]
    Comme $\text{Hom}_{A}(A e,M)$ est un module, par le premier théorème d'isomorphisme, \\
    $\frac{\text{Hom}_{A}(A e,M)}{\text{ker}(\phi)} \cong \text{Im}(\phi)$\\
    Le noyau de $\phi$ est trivial, car un unique homomorphisme peut être défini tq $\sigma(e) = 0$, car tous éléments du domaine sont de la forme ae.\\
    Il ne reste plus qu'a montré que l'image est eM.\\
    Par le lemme HomAeM, on a que $\text{Im}(\phi) \subset e M$.\\
    Soit $m \in e M$. Considérons $\psi(e)=m$ tq $\psi(ae) = am$.\\
    Il s'agit d'un homomorphisme de $A e \to M$.\\
    Donc $\text{Im}(\phi) = e M$
\end{proof}