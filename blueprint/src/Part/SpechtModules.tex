\chapter{SpechtModules}

\begin{definition}[YoungProjectors]
    \label{YoungProjectors}
    \uses{Pu, Qu, PuCard, QuCard}
    Un Young projector est défini par un YoungDiagram $\mu$\\
    \[ a_{\mu} := \frac{1}{|P_{\mu}|}\sum_{g \in P_{\mu}}g \]
    \[ b_{\mu} := \frac{1}{|Q_{\mu}|}\sum_{g \in Q_{\mu}}(-1)^{g}g \]
    Où $(-1)^{g}$ est le signe de g
\end{definition}

\begin{definition}[YoungSymmetriser]
    \label{YoungSymmetriser}
    \uses{YoungProjectors}
    Un Young symmetriser est défini par un YoungDiagram $\mu$\\
    $c_{\mu} := a_{\mu} b_{\mu} $
\end{definition}

\begin{definition}[SpechtModules]
    \label{SpechtModules}
    \uses{YoungSymmetriser}
    Soit $\mu$ un YoungDiagram.
    \[ V_{\mu} := \mathbb{C}[S_{n}]c_{\mu} \]
    $V_{\mu}$ est appelé un Specht modules.\\
    Il est un sous-espace de $\mathbb{C}[S_{n}]$.
\end{definition}

\begin{lemma}[LinearTransformation]
    \label{LinearTransformation}
    \uses{YoungSymmetriser}
    $\exists l_{\mu}$ une fonction linéaire tq\\
    $\forall x \in \mathbb{C}[S_{n}]$, $a_{\mu} x b_{\mu} = l_{\mu}(x) c_{\mu}$
\end{lemma}
\begin{proof}
    \uses{No2FromSameColToSameRow}
    Soit x $\in \mathbb{C}[S_{n}]$.\\
    x est de la forme $\sum_{g \in S_n} a_g g$. Examinons se qu'il se passe pour différent g.\\
    Si g $\in P_{\mu} Q_{\mu}$, alors $\exists$ p $\in P_{\mu}$ et q $\in Q_{\mu}$ tq g=pq\\
    \[ a_{\mu} g b_{\mu} = \frac{1}{|P_{\mu}|}\sum_{g \in P_{\mu}}g \; p q \frac{1}{|Q_{\mu}|}\sum_{h \in Q_{\mu}}(-1)^{h} \]
    \[ \frac{1}{|P_{\mu}|}\sum_{g \in P_{\mu}}g p = \frac{1}{|P_{\mu}|}\sum_{g' \in P_{\mu}}g' \]
    On peut faire le changement de variable en posant $g' = g p$ et en utilisant le fait que $\phi (g) =g p$ est un isomorphisme de groupe. Ainsi les deux sommes sont équivalantes à un réordenement près.
    \[ \frac{1}{|Q_{\mu}|}\sum_{h \in Q_{\mu}}(-1)^{h}q h = \frac{1}{|Q_{\mu}|}\sum_{h \in Q_{\mu}}(-1)^{h}q h = (-1)^{q^{-1}} \frac{1}{|Q_{\mu}|}\sum_{h' \in Q_{\mu}}(-1)^{h'}h' \]
    \[ a_{\mu} g b_{\mu} = (-1)^{q}c_{\mu} \]
    Il ne reste plus à montrer que si g $\notin P_{\mu} Q_{\mu}$ alors $l_{\mu}(g)$=0, car g ne peut pas être exprimer par $c_{\mu}$\\
    Donc il faut mq $a_{\mu} g b_{\mu}$=0 ou de façon équivalente $a_{\mu} g b_{\mu} = -a_{\mu} g b_{\mu}$\\
    Il suffit de trouver $t \in P_{\mu}$ tq $g^{-1} t g \in Q_{\mu}$ et $(-1)^t = -1$, car
    \[ a_{\mu} g b_{\mu} = a_{\mu} t g b_{\mu} = a_{\mu} (g g^{-1}) t g b_{\mu} = a_{\mu} g (g^{-1} t g) b_{\mu} = (-1)^{g^{-1} t g} a_{\mu} g b_{\mu} = -a_{\mu} g b_{\mu} \]
    Plusieurs changements de variables ont été effectuer pour "faire apparaître et disparaître" des éléments. $(-1)^{g^{-1} t g} = (-1)^{g^{-1}}\cdot (-1)^{t}\cdot (-1)^{g} = (-1)^{g}\cdot (-1)^{t}\cdot (-1)^{g} = -1$\\
    Par la contraposé du lemme No2FromSameColToSameRow, on a que\\
    $\exists i,j,k,l \in \mu$ tq $i \neq j, g(Y_{\mu}(i))=Y_{\mu}(k), g(Y_{\mu}(j))=Y_{\mu}(l), i.x=j.x$ et $k.y = l.y$.\\
    Posons t : [0,n-1] $\to$ [0,n-1]
    \[ t(n) = \begin{cases} Y_{\mu}(k) & \text{si } n=Y_{\mu}(l)\\
                            Y_{\mu}(l) & \text{si } n=Y_{\mu}(k)\\
                            n & \text{sinon} 
    \end{cases}\]
    Par construction, $t \in P_{\mu}$ et $(-1)^t = -1$. Il suffit de montré que $g^{-1} t g \in Q_{\mu}$
    \[ g^{-1} \circ t \circ g(Y_{\mu}(i)) = g^{-1} \circ t (Y_{\mu}(k)) = g^{-1}(Y_{\mu}(l)) = Y_{\mu}(j) \]
    \[ g^{-1} \circ t \circ g(Y_{\mu}(j)) = g^{-1} \circ t (Y_{\mu}(l)) = g^{-1}(Y_{\mu}(k)) = Y_{\mu}(i) \]
    On remarque que si $m \in \mu \backslash \{i,j\}, g(Y_{\mu}(m)) \notin \{Y_{\mu}(k), Y_{\mu}(l)\}$. Donc $t(g(Y_{\mu}(m)))$ se comporte comme la fonction identité. Ainsi $g^{-1} t g \in Q_{\mu}$.\\
\end{proof}

\begin{lemma}[SmallerImpZero]
    \label{SmallerImpZero}
    \uses{YoungSymmetriser, IneqYoungDiagram}
    Si $\mu > \lambda$, alors\\
    \[a_{\mu} \mathbb{C}[S_{n}] b_{\lambda} = 0\]
\end{lemma}
\begin{proof}
    Comme $\mu > \lambda$\\
    TODO montré que\\
    Donc, il existe deux éléments de la même colomne que g envoit sur la même rangé\\
    Ainsi un peut construire un t tq t $\in P_{\mu}$ et $g^{-1} t g \in Q_{\lambda}$.\\
    Par le même argument que le dernier lemme, $a_{\mu} \mathbb{C}[S_{n}] b_{\lambda} = 0$
\end{proof}

\begin{lemma}[CuPropIdempotent]
    \label{CuIdempotent}
    \uses{YoungSymmetriser}
    $c_{\mu}$ est proportionel à un idempotent. De façon mathématique
    \[\exists a \in \mathbb{C}, c_{\mu}^{2} = a \cdot c_{\mu} \]
\end{lemma}
\begin{proof}
    \uses{LinearTransformation}
    On applique le lemme LinearTransformation avec $x = b_{\mu} a_{\mu} \in \mathbb{C}[S_{n}]$.
\end{proof}

\begin{theorem}[IrreductibleRepresentationSn]
    \label{IrreductibleRepresentationSn}
    \uses{SpechtModules}
    $\forall \mu$ partition de n, $V_{\mu}$ est toute les représentations irréductibles de $S_n$
\end{theorem}
\begin{proof}
    \uses{HomAeMisoeM, CuPropIdempotent, SmallerImpZero, LinearTransformation}
    Soit $\mu, \lambda$ deux partitions de n et sans perte de généralité, $\mu \geq \lambda$\\
    \[ \text{Hom}_{\mathbb{C}[S_{n}]}(V_{\mu}, V_{\lambda}) = \text{Hom}_{\mathbb{C}[S_{n}]}(\mathbb{C}[S_{n}]c_{\mu}, \mathbb{C}[S_{n}]c_{\lambda}) \cong c_{\mu}\mathbb{C}[S_{n}]c_{\lambda}  \]
    Si $\mu > \lambda$ alors $c_{\mu}\mathbb{C}[S_{n}]c_{\lambda} = 0$\\
    Sinon $\mu = \lambda$ et on a une représentation de dimension 1.\\
    Comme le nombre de partition de n est égale au nombre de classe de conjugaison de $S_n$ on a que tous les représentations de $S_n$ sont atteintes par $V_{\mu}$.
\end{proof}